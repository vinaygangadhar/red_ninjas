\section{Motivation}\label{sec:motivation}


In this project, we intend to implement face detection algorithm based on the Viola Jones classifier \fixme{ref} on GPU. 
As a starting point, we take the GNU licensed C++ program that has the 
algorithm implemented to detect faces in images \fixme{ref}. 
There are various portions in the algorithm that can be parallelized and hence can leverage the hardware of GPU efficiently. 
Section 1 explains Viola Jones algorithm briefly. Section 2 explains the portions we are going to parallelize and offload to the GPU.


Viola Jones Face Detection Algorithm
The algorithm has four stages:
Haar feature selection: The Viola Jones classifier method is based on Haar-like features. 
The features consist of white and black rectangles as shown in Fig 1. 
These features can be thought of as pixel intensity evaluation sets. 
For each feature, we subtract black region’s pixel value sum from white region’s pixel value sum. 
If this sum is greater than some threshold, it is decided that the region has this feature. 
This is the characteristic value of a feature. We have the Haar features
to be used for face detection

