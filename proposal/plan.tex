\section{Project phases}\label{sec:plan}

We explain the phases of project here and expected timeline for 
each step. Some of the phases can be carried out in parallel and will be distributed 
among the project members.

\begin{itemize}
\item \textbf{Phase 0 [1 week]:} Choose the representative workloads/ kernels of neural network domain
targeting different applications (image processing, speech recognition, text parsing etc.,). 
Use the trace based modeling tool TDG (~\cite{tonytdg}) modeled
for PENN to determine the initial performance and power estimates for each kernel.
More accurate analysis of workloads (profiling) to be done here and see if there are any missing architectural
pieces not considered for current PENN architecture.

\item \textbf{Phase 1 [2 days]:} Based on phase 0 analysis, some important design trade-off decisions should be taken. Those steps are listed here:
i) Low-power core: Properties (pipeline stages, memory interface and hierarchy etc.,) of low-power core which does the co-ordination.
Compiler and software toolchain for the core (RISCV toolchain and their in-order core can be a good start). 
ii) CGRA: Types of problem specific FUs inside the fabric.  Interface to low-power core and scratchpad memory.
Addressing of scratchpad space and global memory. Scheduling pattern for CGRA. Scratchpad size and bit-width.
iii) Programming model: API for PENN. Pragmas and code annotations for compilation.

\item \textbf{Phase 2 [3 weeks]:} This phase involves implementation of individual modules of PENN listed in Phase 1. 
Implementation of low power core and CGRA in Verilog and C++.
Writing API routines for PENN and compiler support (Note: A full-fledged compiler may not be implemented but 
a framework to generate instructions and configuration stream will be implemented.)
Tools needed for the project are explained in Section~\ref{sec:meth}.

\item \textbf{Phase 3 [1 week]:}This phase mainly involves evaluating implemented synthesized PENN architecture with
    representative workloads chosen in Phase 0. We also use C++ simulator written for PENN to correlate the functionality of the synthesized version. 
We plan to evaluate PENN for performance, area and power against a state-of-the-art DSA for Neural Network.

\item \textbf{Phase 4 [2 weeks]:} This phase mainly involves prototyping PENN on Zynq FPGA. However, this phase will be realized only if Phase 3 is
    completed well within course project time limit. Project report is also part of this phase.
\end{itemize}




