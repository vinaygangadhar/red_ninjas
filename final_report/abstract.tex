\begin{abstract}
   
\vspace{0.1in}

Applications with large amount of data level parallelism
can benefit from General Purpose based Graphics Processing
Units (GPGPUs) because of better energy efficiency and performance compared to a CPU. Due to GPGPUs’ impressive
computing throughput and memory bandwidth, many applications with enough parallelism can take advantage of acceleration using GPGPU. Computer vision algorithms are one such
workload domain which can better utilize the large number
of cores in GPU, and hence meet their real-time requirements
of the applications. One such application is \emph{Face Detection}
which has real-time constraint on its execution. Sequential
processing of image windows with image downsampling and classifiers on CPU is difficult to meet the real-time requirements. 

In this project, we have
implemented the face detection acceleration algorithm based
on Viola-Jones cascade classifier on the GPGPU CUDA platform.
We have considered different portions of the viola jones algorithm which include 
\emph{nearest neighbor}, \emph{integral image} and {HAAR classifier} that can
be parallelized.
We identify different bottlenecks in GPU implementation and include optimizations
which gives the performance benefit. We explain each of these optimizations in detail for all the kernels. 
We finally compare the execution of the same algorithm executed
on the CPU and analyze the gains from the GPU.
We achieve a speedup upto 5.35x (including the inclusive time) compared to the single threaded performance of CPU.




\end{abstract}
