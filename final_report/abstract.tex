\begin{abstract}
   
\vspace{0.1in}

Applications with large amount of data level parallelism
can benefit from General Purpose based Graphics Processing
Units (GPGPUs) because of better energy efficiency and per-
formance compared to a CPU. Due to GPGPUs’ impressive
computing throughput and memory bandwidth, many applica-
tions with enough parallelism can take advantage of accelera-
tion using GPGPU. Computer vision algorithms are one such
workload domain which can better utilize the large number
of cores in GPU, and hence meet their real-time requirements
of the applications. One such application is Face Detection
which has real-time constraint on its execution. Sequential
processing of image windows with classifiers on CPU is dif-
ficult to meet the real-time requirements. In this project, we
implement the Face Detection acceleration algorithm based
on Viola-Jones cascade classifier the GPGPU CUDA platform.
We are considering different portions of the algorithm that can
be parallelized and expect to achieve better speedup compared.
We feed an image as input to the GPU, and after processing set
of features and classifying the face using a cascade classifier,
a rectangle is drawn on the image when face is detected. We
finally compare the execution of the same algorithm executed
on the CPU and analyze the gains from the GPU.





\end{abstract}
